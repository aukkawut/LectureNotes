\begin{defbox}{Topology}
Let $X$ be a set, with $x \in X$ representing elements or points in $X$. Let $\mathcal{K}$ be a collection of subsets of $X$, satisfying the following properties:
\begin{enumerate}
    \item $\emptyset, X \in \mathcal{K}$,
    \item The intersection of any finite number of sets in $\mathcal{K}$ is also in $\mathcal{K}$,
    \item The union of any collection of sets in $\mathcal{K}$ is also in $\mathcal{K}$.
\end{enumerate}
If these conditions are satisfied, $\mathcal{K}$ defines a \textit{topology} on $X$, and the pair $(X, \mathcal{K})$ is called a \textit{topological space}.
\end{defbox}

A topology provides a framework to study the structure of sets and their properties in an abstract way.
Now, let us consider some additional terms and their implications:

\begin{defbox}{Open and Closed Sets}
    In a topological space $(X, \mathcal{K})$:
\begin{itemize}
    \item The sets in $\mathcal{K}$ are called \textit{open sets}.
    \item A subset $B \subseteq X$ is \textit{closed} if $X \setminus B$ (the complement of $B$ in $X$) is an open set.
\end{itemize}
\end{defbox} 


\begin{exbox}{Topology}
To illustrate these concepts, let us consider two examples:
\begin{enumerate}
    \item \textbf{Trivial Topology:} On any set $X$, the collection $\mathcal{K} = \{ \emptyset, X \}$ forms a topology. The corresponding topological space $(X, \mathcal{K})$ is called the \textit{trivial topology}.
    \item \textbf{Discrete Topology:} For any set $X$, let $\mathcal{K} = 2^X$ (the power set of $X$, i.e., the collection of all subsets of $X$). The resulting topological space $(X, \mathcal{K})$ is called the \textit{discrete topology}, where every subset of $X$ is both open and closed.
\end{enumerate}
\end{exbox}

\subsection{Standard Topology on $\mathbb{R}^n$}


\begin{defbox}{Open Sets in $\mathbb{R}^n$}
Define a collection of subsets of $\mathbb{R}^n$ as:
\[
\mathcal{B}_0 = \left\{ \prod_{i=1}^n (a_i, b_i) \;\middle|\; (a_i, b_i) \subset \mathbb{R}, \; a_i, b_i \in \mathbb{R}, \; i = 1, \dots, n \right\}.
\]
Here, each $(a_i, b_i)$ is an open interval in $\mathbb{R}$.
\end{defbox}
Using this definition on $\mathbb{R}^2$, this becomes:
\[
\mathcal{B}_0 = \{ (a_1, b_1) \times (a_2, b_2) \mid a_i, b_i \in \mathbb{R}, \; i = 1, 2 \}.
\]
\subsubsection{Constructing the Topology}
So, first problem we can see is that $\mathcal{B}_0$ is not topology. To fix that, we can construct it in different way:
\begin{enumerate}
    \item \textbf{Finite Intersections:}
    Construct $\mathcal{F}_1$ by including all finite intersections of sets in $\mathcal{B}_0$:
    \[
    \mathcal{F}_1 = \mathcal{B}_0 \cup \bigcap_{i=1}^k A_i, \quad A_i \in \mathcal{B}_0.
    \]
    The collection $\mathcal{F}_1$ is stable under finite intersections.

    \item \textbf{Arbitrary Unions:}
    Extend $\mathcal{F}_1$ to include arbitrary unions of sets. Let:
    \[
    \mathcal{F}_2 = \mathcal{F}_1 \cup \bigcup_{A_i \in \mathcal{F}_1} A_i.
    \]
    The collection $\mathcal{F}_2$ is stable under both finite intersections and arbitrary unions.
\end{enumerate}

Thus, $\mathcal{F}_2$ defines a topology on $\mathbb{R}^n$. The topological space $(\mathbb{R}^n, \mathcal{F}_2)$ is called the \textit{standard topology} on $\mathbb{R}^n$. However, for the sake of clear notation, $\Phi^*$ is the standard topology.

\subsection{Topology on a Finite Set}

Let $X = \{ a, b \}$. Define:
\[
\mathcal{K} = \{ \emptyset, \{ a \}, \{ b \}, \{ a, b \} \}.
\]
The pair $(X, \mathcal{K})$ is a topological space. Observe:
\begin{itemize}
    \item $\{ b \}$ is a closed set.
    \item $\{ a \}$ is an open set but not closed.
\end{itemize}
We will revisit this example later.
\subsection{Separation Properties in Topological Spaces}
\begin{defbox}{Hausdorff Space}
A topological space $(X, \mathcal{K})$ is called \textit{Hausdorff} if, for any two distinct points $x, y \in X$, there exist disjoint open sets $A, B \in \mathcal{K}$ such that:
\[
    x \in A \quad \text{and} \quad y \in B, \quad \text{with} \quad A \cap B = \emptyset.
\]
\end{defbox}
This property ensures that distinct points can be "separated" by open neighborhoods.

\begin{defbox}{Normal Space}
A topological space $(X, \mathcal{K})$ is called \textit{normal} if, for any two disjoint closed sets $E$ and $F$, there exist disjoint open sets $A, B \in \mathcal{K}$ such that:
\[
    E \subseteq A, \quad F \subseteq B, \quad \text{and} \quad A \cap B = \emptyset.
\]
\end{defbox}
This property extends the concept of separation to closed sets. Looking back at our previous example,

\begin{enumerate}
    \item \textbf{Standard Topology on $\mathbb{R}^n$:}
    \begin{itemize}
        \item With the standard topology, $\mathbb{R}^n$ is both Hausdorff and normal.
        \item For any two disjoint closed intervals, such as $E = (-\infty, 3]$ and $F = [5, \infty)$, disjoint open sets can be constructed to separate them.
    \end{itemize}

    \item \textbf{Topology on a Finite Set:}
    \begin{itemize}
        \item Let $X = \{ a, b \}$ and $\mathcal{K} = \{ \emptyset, \{ a \}, \{ b \}, \{ a, b \} \}$.
        \item \textbf{Question:} Is $(X, \mathcal{K})$ Hausdorff? \textbf{Answer:} No, $X$ is not Hausdorff since $a$ and $b$ cannot be separated by disjoint open sets.
        \item \textbf{Question:} Is $(X, \mathcal{K})$ normal? \textbf{Answer:} No, $X$ is not normal for the same reason.
    \end{itemize}
\end{enumerate}
\subsection{Neighborhood}
\begin{defbox}{Neighborhood}
A \textit{neighborhood} of a set $B \subseteq X$ in a topological space is any open set $A$ such that $B \subseteq A$.
\end{defbox}
\begin{exbox}{Neighborhood}

Let $B = (1, 2) \cup (4, 5)$. Then $A = (-\infty, 0) \cup (3, \infty)$ is not a neighborhood of $B$ because $A$ does not contain $B$.

\end{exbox}
\begin{defbox}{Closure}
The \textit{closure} of a set $B \subseteq X$, denoted $\overline{B}$, is the smallest closed set containing $B$:
\[
\overline{B} = \bigcap_{B \subseteq A, A \text{ closed}} A.
\]
This is the intersection of all closed sets containing $B$.
\end{defbox}
\begin{defbox}{Interior}
The \textit{interior} of a set $B$, denoted $B^\circ$ or $\text{int}(B)$, is the largest open set contained in $B$:
\[
B^\circ = \bigcup_{A \subseteq B, A \text{ open}} A.
\]
\end{defbox}
\begin{defbox}{Open Covering}
An \textit{open covering} of a set $B \subseteq X$ is a collection of open sets $\{A_i\}_{i \in I}$ such that:
\[
B \subseteq \bigcup_{i \in I} A_i.
\]
\end{defbox}
\begin{defbox}{Compactness}
A set $B \subseteq X$ is called \textit{compact} if every open covering of $B$ contains a finite subcollection that also covers $B$.
\end{defbox}
\subsection{Convergence and Continuity}
\subsubsection{Problem with Convergence}

Recall that in real analysis, we have Heine-Borel theorem
\begin{thmbox}{Heine-Borel}
For  $S\subseteq \mathbb{R}^n$ then $S$ is compact iff $S$ is closed and bounded.
\end{thmbox}
However, says we have topological space $(\mathbb{R}^n,\Phi^*)$. What is "bounded" in this space sense? In real analysis, we can say that bounded then it must have limit to something but to have a limit you need convergence. Say we have a sequence of element in a topological space $\{x_i\}_{i\geq 1}$, can we say anything about it? What does it means to have $x_k \rightarrow x_0$ in this sense?
\subsubsection{Convergence in Topological Space}
\begin{defbox}{Convergence}
    Let $(X, \mathcal{K})$ be a topological space. We say that a sequence $\{x_n\}_{n \geq 1} \subseteq X$ converges to a point $x_0 \in X$ if, for every open set $U \in \mathcal{K}$ containing $x_0$, there exists $N \in \mathbb{N}$ such that $x_n \in U$ for all $n \geq N$.

\[
\forall U \in \mathcal{K}, \, x_0 \in U \implies \exists N \in \mathbb{N} \text{ such that } x_n \in U \, \forall n \geq N.
\]
\end{defbox}
\begin{exbox}{Convergence on Topology of Finite Set}
    Consider the topological space $X = \{a, b\}$ with the topology $\mathcal{K} = \{\emptyset, \{a\}, \{a, b\}\}$. 

\begin{itemize}
    \item The sequence $\{a, b, a, b, \dots\}$ converges to $b$ (by definition).

    Here, we want to see whether it will converge to $b$, we first can notice that the only open subset in the topology containing $b$ is $\{a,b\}$. Now, the question is will it eventually lie in that set? Answer is obviously.
    \item The constant sequence $\{a, a, a, \dots\}$ converges to $a$, and also to $b$.
    
    Let's consider the first case: $a$. So, there are two open sets $\{a\}, \{a,b\}$. The sequence will eventually be $a$ which is in both sets. By definition, this converges to $a$. In the second case, there is only one open set in the topology that contains $b$ which is $\{a,b\}$. This sequence will end with $a$ which is in that set, hence it is converging to $b$.
\end{itemize}
Note that convergence in topological space might not be unique.
\end{exbox}
\begin{thmbox}{Uniqueness of Limits in Hausdorff Spaces}
If $X$ is a Hausdorff space, then every convergent sequence in $X$ has a unique limit.
\end{thmbox}
\begin{proof}
    Suppose there exist $\{x_n\}_{n\geq 1}$ that does not converge to unique limit; say converges to $x$ and $y$ ($x\neq y$) and let $X$ be Hausdorff.
    
    Using the fact that $X$ is Hausdorff, we can say that there exists the open and disjoint set $U$ and $V$ in the topology of $X$ such that $x\in U$ and $y\in V$.

    Then since $x_n \rightarrow x$ and $x_n\rightarrow y$ then there exists $N_1\in\mathbb{N}$ such that $x_{p} \in U,\; p > N_1$. Likewise, there also exists $N_2 \in \mathbb{N}$ such that $x_{q} \in V, q > N_2$.

    However, since $X$ is Hausdorff, this is impossible as $U$ and $V$ are disjoint. Hence, our hypothesis is false. That means, if $X$ is Hausdorff, convergent sequence can only have unique limit
\end{proof} 
\subsubsection{Continuity}
\begin{defbox}{Continuous Function}
    Let $f : X \to Y$ be a function between topological spaces. We say that $f$ is \textit{continuous} if, for any open set $U \subseteq Y$, the preimage $f^{-1}(U)$ is open in $X$. That is,
    \[
f \text{ is continuous} \iff \forall U \subseteq Y, \, U \text{ open} \implies f^{-1}(U) \text{ is open in } X.
\]
\end{defbox}
From this definition, we can see that the following theorem comes natural with it
\begin{thmbox}{Convergence under Continuous Function in Topological Space}
    Suppose $f: X \rightarrow Y$ be a continuous function between topological spaces. Let $\{x_n\}_{n\geq 1}$ be a sequence from $X$ such that it converges to $x$. Then,
    \[
    x_n \rightarrow x \implies f(x_n) \rightarrow f(x)
    \]
\end{thmbox}
\begin{proof}
    Assume $f:X\rightarrow Y$ is continuous and assume $x_n \rightarrow x\in X$. We need to show that $f(x_n)\rightarrow f(x)$. 

    Let $U$ be an open set in $Y$ that contains $f(x)$. Since $f$ is continuous,
    \[
    f(x)\in U \iff x \in f^{-1}(U) \text{ is open in }X
    \]
    Then, there exists $N_0\in\mathbb{N}$ such that for all $n > N_0$,
    \[
    x_n \in f^{-1}(U) \iff f(x_n) \in U
    \]
    Therefore,
    \[
    x_n \rightarrow x \implies f(x_n) \rightarrow f(x)
    \]
\end{proof}
